
% Default to the notebook output style

    


% Inherit from the specified cell style.




    
\documentclass[11pt]{article}

    
    
    \usepackage[T1]{fontenc}
    % Nicer default font (+ math font) than Computer Modern for most use cases
    \usepackage{mathpazo}

    % Basic figure setup, for now with no caption control since it's done
    % automatically by Pandoc (which extracts ![](path) syntax from Markdown).
    \usepackage{graphicx}
    % We will generate all images so they have a width \maxwidth. This means
    % that they will get their normal width if they fit onto the page, but
    % are scaled down if they would overflow the margins.
    \makeatletter
    \def\maxwidth{\ifdim\Gin@nat@width>\linewidth\linewidth
    \else\Gin@nat@width\fi}
    \makeatother
    \let\Oldincludegraphics\includegraphics
    % Set max figure width to be 80% of text width, for now hardcoded.
    \renewcommand{\includegraphics}[1]{\Oldincludegraphics[width=.8\maxwidth]{#1}}
    % Ensure that by default, figures have no caption (until we provide a
    % proper Figure object with a Caption API and a way to capture that
    % in the conversion process - todo).
    \usepackage{caption}
    \DeclareCaptionLabelFormat{nolabel}{}
    \captionsetup{labelformat=nolabel}

    \usepackage{adjustbox} % Used to constrain images to a maximum size 
    \usepackage{xcolor} % Allow colors to be defined
    \usepackage{enumerate} % Needed for markdown enumerations to work
    \usepackage{geometry} % Used to adjust the document margins
    \usepackage{amsmath} % Equations
    \usepackage{amssymb} % Equations
    \usepackage{textcomp} % defines textquotesingle
    % Hack from http://tex.stackexchange.com/a/47451/13684:
    \AtBeginDocument{%
        \def\PYZsq{\textquotesingle}% Upright quotes in Pygmentized code
    }
    \usepackage{upquote} % Upright quotes for verbatim code
    \usepackage{eurosym} % defines \euro
    \usepackage[mathletters]{ucs} % Extended unicode (utf-8) support
    \usepackage[utf8x]{inputenc} % Allow utf-8 characters in the tex document
    \usepackage{fancyvrb} % verbatim replacement that allows latex
    \usepackage{grffile} % extends the file name processing of package graphics 
                         % to support a larger range 
    % The hyperref package gives us a pdf with properly built
    % internal navigation ('pdf bookmarks' for the table of contents,
    % internal cross-reference links, web links for URLs, etc.)
    \usepackage{hyperref}
    \usepackage{longtable} % longtable support required by pandoc >1.10
    \usepackage{booktabs}  % table support for pandoc > 1.12.2
    \usepackage[inline]{enumitem} % IRkernel/repr support (it uses the enumerate* environment)
    \usepackage[normalem]{ulem} % ulem is needed to support strikethroughs (\sout)
                                % normalem makes italics be italics, not underlines
    

    
    
    % Colors for the hyperref package
    \definecolor{urlcolor}{rgb}{0,.145,.698}
    \definecolor{linkcolor}{rgb}{.71,0.21,0.01}
    \definecolor{citecolor}{rgb}{.12,.54,.11}

    % ANSI colors
    \definecolor{ansi-black}{HTML}{3E424D}
    \definecolor{ansi-black-intense}{HTML}{282C36}
    \definecolor{ansi-red}{HTML}{E75C58}
    \definecolor{ansi-red-intense}{HTML}{B22B31}
    \definecolor{ansi-green}{HTML}{00A250}
    \definecolor{ansi-green-intense}{HTML}{007427}
    \definecolor{ansi-yellow}{HTML}{DDB62B}
    \definecolor{ansi-yellow-intense}{HTML}{B27D12}
    \definecolor{ansi-blue}{HTML}{208FFB}
    \definecolor{ansi-blue-intense}{HTML}{0065CA}
    \definecolor{ansi-magenta}{HTML}{D160C4}
    \definecolor{ansi-magenta-intense}{HTML}{A03196}
    \definecolor{ansi-cyan}{HTML}{60C6C8}
    \definecolor{ansi-cyan-intense}{HTML}{258F8F}
    \definecolor{ansi-white}{HTML}{C5C1B4}
    \definecolor{ansi-white-intense}{HTML}{A1A6B2}

    % commands and environments needed by pandoc snippets
    % extracted from the output of `pandoc -s`
    \providecommand{\tightlist}{%
      \setlength{\itemsep}{0pt}\setlength{\parskip}{0pt}}
    \DefineVerbatimEnvironment{Highlighting}{Verbatim}{commandchars=\\\{\}}
    % Add ',fontsize=\small' for more characters per line
    \newenvironment{Shaded}{}{}
    \newcommand{\KeywordTok}[1]{\textcolor[rgb]{0.00,0.44,0.13}{\textbf{{#1}}}}
    \newcommand{\DataTypeTok}[1]{\textcolor[rgb]{0.56,0.13,0.00}{{#1}}}
    \newcommand{\DecValTok}[1]{\textcolor[rgb]{0.25,0.63,0.44}{{#1}}}
    \newcommand{\BaseNTok}[1]{\textcolor[rgb]{0.25,0.63,0.44}{{#1}}}
    \newcommand{\FloatTok}[1]{\textcolor[rgb]{0.25,0.63,0.44}{{#1}}}
    \newcommand{\CharTok}[1]{\textcolor[rgb]{0.25,0.44,0.63}{{#1}}}
    \newcommand{\StringTok}[1]{\textcolor[rgb]{0.25,0.44,0.63}{{#1}}}
    \newcommand{\CommentTok}[1]{\textcolor[rgb]{0.38,0.63,0.69}{\textit{{#1}}}}
    \newcommand{\OtherTok}[1]{\textcolor[rgb]{0.00,0.44,0.13}{{#1}}}
    \newcommand{\AlertTok}[1]{\textcolor[rgb]{1.00,0.00,0.00}{\textbf{{#1}}}}
    \newcommand{\FunctionTok}[1]{\textcolor[rgb]{0.02,0.16,0.49}{{#1}}}
    \newcommand{\RegionMarkerTok}[1]{{#1}}
    \newcommand{\ErrorTok}[1]{\textcolor[rgb]{1.00,0.00,0.00}{\textbf{{#1}}}}
    \newcommand{\NormalTok}[1]{{#1}}
    
    % Additional commands for more recent versions of Pandoc
    \newcommand{\ConstantTok}[1]{\textcolor[rgb]{0.53,0.00,0.00}{{#1}}}
    \newcommand{\SpecialCharTok}[1]{\textcolor[rgb]{0.25,0.44,0.63}{{#1}}}
    \newcommand{\VerbatimStringTok}[1]{\textcolor[rgb]{0.25,0.44,0.63}{{#1}}}
    \newcommand{\SpecialStringTok}[1]{\textcolor[rgb]{0.73,0.40,0.53}{{#1}}}
    \newcommand{\ImportTok}[1]{{#1}}
    \newcommand{\DocumentationTok}[1]{\textcolor[rgb]{0.73,0.13,0.13}{\textit{{#1}}}}
    \newcommand{\AnnotationTok}[1]{\textcolor[rgb]{0.38,0.63,0.69}{\textbf{\textit{{#1}}}}}
    \newcommand{\CommentVarTok}[1]{\textcolor[rgb]{0.38,0.63,0.69}{\textbf{\textit{{#1}}}}}
    \newcommand{\VariableTok}[1]{\textcolor[rgb]{0.10,0.09,0.49}{{#1}}}
    \newcommand{\ControlFlowTok}[1]{\textcolor[rgb]{0.00,0.44,0.13}{\textbf{{#1}}}}
    \newcommand{\OperatorTok}[1]{\textcolor[rgb]{0.40,0.40,0.40}{{#1}}}
    \newcommand{\BuiltInTok}[1]{{#1}}
    \newcommand{\ExtensionTok}[1]{{#1}}
    \newcommand{\PreprocessorTok}[1]{\textcolor[rgb]{0.74,0.48,0.00}{{#1}}}
    \newcommand{\AttributeTok}[1]{\textcolor[rgb]{0.49,0.56,0.16}{{#1}}}
    \newcommand{\InformationTok}[1]{\textcolor[rgb]{0.38,0.63,0.69}{\textbf{\textit{{#1}}}}}
    \newcommand{\WarningTok}[1]{\textcolor[rgb]{0.38,0.63,0.69}{\textbf{\textit{{#1}}}}}
    
    
    % Define a nice break command that doesn't care if a line doesn't already
    % exist.
    \def\br{\hspace*{\fill} \\* }
    % Math Jax compatability definitions
    \def\gt{>}
    \def\lt{<}
    % Document parameters
    \title{56301 Cufaro G Idea TPFinal}
    
    
    

    % Pygments definitions
    
\makeatletter
\def\PY@reset{\let\PY@it=\relax \let\PY@bf=\relax%
    \let\PY@ul=\relax \let\PY@tc=\relax%
    \let\PY@bc=\relax \let\PY@ff=\relax}
\def\PY@tok#1{\csname PY@tok@#1\endcsname}
\def\PY@toks#1+{\ifx\relax#1\empty\else%
    \PY@tok{#1}\expandafter\PY@toks\fi}
\def\PY@do#1{\PY@bc{\PY@tc{\PY@ul{%
    \PY@it{\PY@bf{\PY@ff{#1}}}}}}}
\def\PY#1#2{\PY@reset\PY@toks#1+\relax+\PY@do{#2}}

\expandafter\def\csname PY@tok@w\endcsname{\def\PY@tc##1{\textcolor[rgb]{0.73,0.73,0.73}{##1}}}
\expandafter\def\csname PY@tok@c\endcsname{\let\PY@it=\textit\def\PY@tc##1{\textcolor[rgb]{0.25,0.50,0.50}{##1}}}
\expandafter\def\csname PY@tok@cp\endcsname{\def\PY@tc##1{\textcolor[rgb]{0.74,0.48,0.00}{##1}}}
\expandafter\def\csname PY@tok@k\endcsname{\let\PY@bf=\textbf\def\PY@tc##1{\textcolor[rgb]{0.00,0.50,0.00}{##1}}}
\expandafter\def\csname PY@tok@kp\endcsname{\def\PY@tc##1{\textcolor[rgb]{0.00,0.50,0.00}{##1}}}
\expandafter\def\csname PY@tok@kt\endcsname{\def\PY@tc##1{\textcolor[rgb]{0.69,0.00,0.25}{##1}}}
\expandafter\def\csname PY@tok@o\endcsname{\def\PY@tc##1{\textcolor[rgb]{0.40,0.40,0.40}{##1}}}
\expandafter\def\csname PY@tok@ow\endcsname{\let\PY@bf=\textbf\def\PY@tc##1{\textcolor[rgb]{0.67,0.13,1.00}{##1}}}
\expandafter\def\csname PY@tok@nb\endcsname{\def\PY@tc##1{\textcolor[rgb]{0.00,0.50,0.00}{##1}}}
\expandafter\def\csname PY@tok@nf\endcsname{\def\PY@tc##1{\textcolor[rgb]{0.00,0.00,1.00}{##1}}}
\expandafter\def\csname PY@tok@nc\endcsname{\let\PY@bf=\textbf\def\PY@tc##1{\textcolor[rgb]{0.00,0.00,1.00}{##1}}}
\expandafter\def\csname PY@tok@nn\endcsname{\let\PY@bf=\textbf\def\PY@tc##1{\textcolor[rgb]{0.00,0.00,1.00}{##1}}}
\expandafter\def\csname PY@tok@ne\endcsname{\let\PY@bf=\textbf\def\PY@tc##1{\textcolor[rgb]{0.82,0.25,0.23}{##1}}}
\expandafter\def\csname PY@tok@nv\endcsname{\def\PY@tc##1{\textcolor[rgb]{0.10,0.09,0.49}{##1}}}
\expandafter\def\csname PY@tok@no\endcsname{\def\PY@tc##1{\textcolor[rgb]{0.53,0.00,0.00}{##1}}}
\expandafter\def\csname PY@tok@nl\endcsname{\def\PY@tc##1{\textcolor[rgb]{0.63,0.63,0.00}{##1}}}
\expandafter\def\csname PY@tok@ni\endcsname{\let\PY@bf=\textbf\def\PY@tc##1{\textcolor[rgb]{0.60,0.60,0.60}{##1}}}
\expandafter\def\csname PY@tok@na\endcsname{\def\PY@tc##1{\textcolor[rgb]{0.49,0.56,0.16}{##1}}}
\expandafter\def\csname PY@tok@nt\endcsname{\let\PY@bf=\textbf\def\PY@tc##1{\textcolor[rgb]{0.00,0.50,0.00}{##1}}}
\expandafter\def\csname PY@tok@nd\endcsname{\def\PY@tc##1{\textcolor[rgb]{0.67,0.13,1.00}{##1}}}
\expandafter\def\csname PY@tok@s\endcsname{\def\PY@tc##1{\textcolor[rgb]{0.73,0.13,0.13}{##1}}}
\expandafter\def\csname PY@tok@sd\endcsname{\let\PY@it=\textit\def\PY@tc##1{\textcolor[rgb]{0.73,0.13,0.13}{##1}}}
\expandafter\def\csname PY@tok@si\endcsname{\let\PY@bf=\textbf\def\PY@tc##1{\textcolor[rgb]{0.73,0.40,0.53}{##1}}}
\expandafter\def\csname PY@tok@se\endcsname{\let\PY@bf=\textbf\def\PY@tc##1{\textcolor[rgb]{0.73,0.40,0.13}{##1}}}
\expandafter\def\csname PY@tok@sr\endcsname{\def\PY@tc##1{\textcolor[rgb]{0.73,0.40,0.53}{##1}}}
\expandafter\def\csname PY@tok@ss\endcsname{\def\PY@tc##1{\textcolor[rgb]{0.10,0.09,0.49}{##1}}}
\expandafter\def\csname PY@tok@sx\endcsname{\def\PY@tc##1{\textcolor[rgb]{0.00,0.50,0.00}{##1}}}
\expandafter\def\csname PY@tok@m\endcsname{\def\PY@tc##1{\textcolor[rgb]{0.40,0.40,0.40}{##1}}}
\expandafter\def\csname PY@tok@gh\endcsname{\let\PY@bf=\textbf\def\PY@tc##1{\textcolor[rgb]{0.00,0.00,0.50}{##1}}}
\expandafter\def\csname PY@tok@gu\endcsname{\let\PY@bf=\textbf\def\PY@tc##1{\textcolor[rgb]{0.50,0.00,0.50}{##1}}}
\expandafter\def\csname PY@tok@gd\endcsname{\def\PY@tc##1{\textcolor[rgb]{0.63,0.00,0.00}{##1}}}
\expandafter\def\csname PY@tok@gi\endcsname{\def\PY@tc##1{\textcolor[rgb]{0.00,0.63,0.00}{##1}}}
\expandafter\def\csname PY@tok@gr\endcsname{\def\PY@tc##1{\textcolor[rgb]{1.00,0.00,0.00}{##1}}}
\expandafter\def\csname PY@tok@ge\endcsname{\let\PY@it=\textit}
\expandafter\def\csname PY@tok@gs\endcsname{\let\PY@bf=\textbf}
\expandafter\def\csname PY@tok@gp\endcsname{\let\PY@bf=\textbf\def\PY@tc##1{\textcolor[rgb]{0.00,0.00,0.50}{##1}}}
\expandafter\def\csname PY@tok@go\endcsname{\def\PY@tc##1{\textcolor[rgb]{0.53,0.53,0.53}{##1}}}
\expandafter\def\csname PY@tok@gt\endcsname{\def\PY@tc##1{\textcolor[rgb]{0.00,0.27,0.87}{##1}}}
\expandafter\def\csname PY@tok@err\endcsname{\def\PY@bc##1{\setlength{\fboxsep}{0pt}\fcolorbox[rgb]{1.00,0.00,0.00}{1,1,1}{\strut ##1}}}
\expandafter\def\csname PY@tok@kc\endcsname{\let\PY@bf=\textbf\def\PY@tc##1{\textcolor[rgb]{0.00,0.50,0.00}{##1}}}
\expandafter\def\csname PY@tok@kd\endcsname{\let\PY@bf=\textbf\def\PY@tc##1{\textcolor[rgb]{0.00,0.50,0.00}{##1}}}
\expandafter\def\csname PY@tok@kn\endcsname{\let\PY@bf=\textbf\def\PY@tc##1{\textcolor[rgb]{0.00,0.50,0.00}{##1}}}
\expandafter\def\csname PY@tok@kr\endcsname{\let\PY@bf=\textbf\def\PY@tc##1{\textcolor[rgb]{0.00,0.50,0.00}{##1}}}
\expandafter\def\csname PY@tok@bp\endcsname{\def\PY@tc##1{\textcolor[rgb]{0.00,0.50,0.00}{##1}}}
\expandafter\def\csname PY@tok@fm\endcsname{\def\PY@tc##1{\textcolor[rgb]{0.00,0.00,1.00}{##1}}}
\expandafter\def\csname PY@tok@vc\endcsname{\def\PY@tc##1{\textcolor[rgb]{0.10,0.09,0.49}{##1}}}
\expandafter\def\csname PY@tok@vg\endcsname{\def\PY@tc##1{\textcolor[rgb]{0.10,0.09,0.49}{##1}}}
\expandafter\def\csname PY@tok@vi\endcsname{\def\PY@tc##1{\textcolor[rgb]{0.10,0.09,0.49}{##1}}}
\expandafter\def\csname PY@tok@vm\endcsname{\def\PY@tc##1{\textcolor[rgb]{0.10,0.09,0.49}{##1}}}
\expandafter\def\csname PY@tok@sa\endcsname{\def\PY@tc##1{\textcolor[rgb]{0.73,0.13,0.13}{##1}}}
\expandafter\def\csname PY@tok@sb\endcsname{\def\PY@tc##1{\textcolor[rgb]{0.73,0.13,0.13}{##1}}}
\expandafter\def\csname PY@tok@sc\endcsname{\def\PY@tc##1{\textcolor[rgb]{0.73,0.13,0.13}{##1}}}
\expandafter\def\csname PY@tok@dl\endcsname{\def\PY@tc##1{\textcolor[rgb]{0.73,0.13,0.13}{##1}}}
\expandafter\def\csname PY@tok@s2\endcsname{\def\PY@tc##1{\textcolor[rgb]{0.73,0.13,0.13}{##1}}}
\expandafter\def\csname PY@tok@sh\endcsname{\def\PY@tc##1{\textcolor[rgb]{0.73,0.13,0.13}{##1}}}
\expandafter\def\csname PY@tok@s1\endcsname{\def\PY@tc##1{\textcolor[rgb]{0.73,0.13,0.13}{##1}}}
\expandafter\def\csname PY@tok@mb\endcsname{\def\PY@tc##1{\textcolor[rgb]{0.40,0.40,0.40}{##1}}}
\expandafter\def\csname PY@tok@mf\endcsname{\def\PY@tc##1{\textcolor[rgb]{0.40,0.40,0.40}{##1}}}
\expandafter\def\csname PY@tok@mh\endcsname{\def\PY@tc##1{\textcolor[rgb]{0.40,0.40,0.40}{##1}}}
\expandafter\def\csname PY@tok@mi\endcsname{\def\PY@tc##1{\textcolor[rgb]{0.40,0.40,0.40}{##1}}}
\expandafter\def\csname PY@tok@il\endcsname{\def\PY@tc##1{\textcolor[rgb]{0.40,0.40,0.40}{##1}}}
\expandafter\def\csname PY@tok@mo\endcsname{\def\PY@tc##1{\textcolor[rgb]{0.40,0.40,0.40}{##1}}}
\expandafter\def\csname PY@tok@ch\endcsname{\let\PY@it=\textit\def\PY@tc##1{\textcolor[rgb]{0.25,0.50,0.50}{##1}}}
\expandafter\def\csname PY@tok@cm\endcsname{\let\PY@it=\textit\def\PY@tc##1{\textcolor[rgb]{0.25,0.50,0.50}{##1}}}
\expandafter\def\csname PY@tok@cpf\endcsname{\let\PY@it=\textit\def\PY@tc##1{\textcolor[rgb]{0.25,0.50,0.50}{##1}}}
\expandafter\def\csname PY@tok@c1\endcsname{\let\PY@it=\textit\def\PY@tc##1{\textcolor[rgb]{0.25,0.50,0.50}{##1}}}
\expandafter\def\csname PY@tok@cs\endcsname{\let\PY@it=\textit\def\PY@tc##1{\textcolor[rgb]{0.25,0.50,0.50}{##1}}}

\def\PYZbs{\char`\\}
\def\PYZus{\char`\_}
\def\PYZob{\char`\{}
\def\PYZcb{\char`\}}
\def\PYZca{\char`\^}
\def\PYZam{\char`\&}
\def\PYZlt{\char`\<}
\def\PYZgt{\char`\>}
\def\PYZsh{\char`\#}
\def\PYZpc{\char`\%}
\def\PYZdl{\char`\$}
\def\PYZhy{\char`\-}
\def\PYZsq{\char`\'}
\def\PYZdq{\char`\"}
\def\PYZti{\char`\~}
% for compatibility with earlier versions
\def\PYZat{@}
\def\PYZlb{[}
\def\PYZrb{]}
\makeatother


    % Exact colors from NB
    \definecolor{incolor}{rgb}{0.0, 0.0, 0.5}
    \definecolor{outcolor}{rgb}{0.545, 0.0, 0.0}



    
    % Prevent overflowing lines due to hard-to-break entities
    \sloppy 
    % Setup hyperref package
    \hypersetup{
      breaklinks=true,  % so long urls are correctly broken across lines
      colorlinks=true,
      urlcolor=urlcolor,
      linkcolor=linkcolor,
      citecolor=citecolor,
      }
    % Slightly bigger margins than the latex defaults
    
    \geometry{verbose,tmargin=1in,bmargin=1in,lmargin=1in,rmargin=1in}
    
    

    \begin{document}
    
    
    \maketitle
    
    

    
    \subsection{Idea Trabajo Práctico Final: Predicción de Resultado de
Partidos de
Tenis}\label{idea-trabajo-pruxe1ctico-final-predicciuxf3n-de-resultado-de-partidos-de-tenis}

El objetivo de la red neuronal a implementar es poder predecir los
resultados de cierta temporada de tenis actual considerando el historial
de partidos de temporadas pasadas, como así también el historial
individual entre los jugadores de cada partido. \#\#\# Dataset de
Referencia El dataset de referencia será el historial de partidos de la
Asociación de Tenistas Profesionales (ATP) brindado en su base de datos.
Este no sólo incluye quien ha ganado cada partido, sino el puntaje y los
porcentajes de las características relevantes de cada jugada en
particular ('primeros servicios', 'segundos servicios', 'aces',
'porcentaje de devoluciones', 'doble faltas', etc.). Por lo tanto, el
set de entrenamiento de la red consistirá de un vector de
características de entrada (X) representando las características de los
jugadores involucrados y del partido, y un vector de salida
correspondiente que determina que jugador gana. A continuación se
realiza el parseo de datos:

    \begin{Verbatim}[commandchars=\\\{\}]
{\color{incolor}In [{\color{incolor}8}]:} \PY{k+kn}{import} \PY{n+nn}{numpy} \PY{k}{as} \PY{n+nn}{np}
        
        \PY{k+kn}{from} \PY{n+nn}{numpy} \PY{k}{import} \PY{n}{genfromtxt}
        \PY{n}{my\PYZus{}data\PYZus{}stats} \PY{o}{=} \PY{n}{genfromtxt}\PY{p}{(}\PY{l+s+s1}{\PYZsq{}}\PY{l+s+s1}{data/match\PYZus{}stats.csv}\PY{l+s+s1}{\PYZsq{}}\PY{p}{,} \PY{n}{delimiter}\PY{o}{=}\PY{l+s+s1}{\PYZsq{}}\PY{l+s+s1}{,}\PY{l+s+s1}{\PYZsq{}}\PY{p}{,}\PY{n}{dtype}\PY{o}{=}\PY{k+kc}{None}\PY{p}{)}
        \PY{n}{my\PYZus{}data\PYZus{}scores} \PY{o}{=} \PY{n}{genfromtxt}\PY{p}{(}\PY{l+s+s1}{\PYZsq{}}\PY{l+s+s1}{data/match\PYZus{}scores.csv}\PY{l+s+s1}{\PYZsq{}}\PY{p}{,} \PY{n}{delimiter}\PY{o}{=}\PY{l+s+s1}{\PYZsq{}}\PY{l+s+s1}{,}\PY{l+s+s1}{\PYZsq{}}\PY{p}{,}\PY{n}{dtype}\PY{o}{=}\PY{k+kc}{None}\PY{p}{)}
\end{Verbatim}


    \begin{Verbatim}[commandchars=\\\{\}]
C:\textbackslash{}Users\textbackslash{}cufar\textbackslash{}Anaconda3\textbackslash{}lib\textbackslash{}site-packages\textbackslash{}ipykernel\_launcher.py:2: VisibleDeprecationWarning: Reading unicode strings without specifying the encoding argument is deprecated. Set the encoding, use None for the system default.
  
C:\textbackslash{}Users\textbackslash{}cufar\textbackslash{}Anaconda3\textbackslash{}lib\textbackslash{}site-packages\textbackslash{}ipykernel\_launcher.py:3: VisibleDeprecationWarning: Reading unicode strings without specifying the encoding argument is deprecated. Set the encoding, use None for the system default.
  This is separate from the ipykernel package so we can avoid doing imports until

    \end{Verbatim}

    \begin{Verbatim}[commandchars=\\\{\}]
{\color{incolor}In [{\color{incolor}39}]:} \PY{c+c1}{\PYZsh{}labels de matchscore}
         
         \PY{n}{tourney\PYZus{}year\PYZus{}id} \PY{o}{=} \PY{l+m+mi}{0}
         \PY{n}{tourney\PYZus{}order} \PY{o}{=} \PY{l+m+mi}{1}
         \PY{n}{tourney\PYZus{}slug} \PY{o}{=} \PY{l+m+mi}{2}
         \PY{n}{tourney\PYZus{}url\PYZus{}suffix} \PY{o}{=} \PY{l+m+mi}{3}
         \PY{n}{tourney\PYZus{}round\PYZus{}name} \PY{o}{=} \PY{l+m+mi}{4}
         \PY{n}{round\PYZus{}order} \PY{o}{=} \PY{l+m+mi}{5}
         \PY{n}{match\PYZus{}order} \PY{o}{=} \PY{l+m+mi}{6}
         \PY{n}{winner\PYZus{}name} \PY{o}{=} \PY{l+m+mi}{7}
         \PY{n}{winner\PYZus{}player\PYZus{}id} \PY{o}{=} \PY{l+m+mi}{8}
         \PY{n}{winner\PYZus{}slug} \PY{o}{=} \PY{l+m+mi}{9}
         \PY{n}{loser\PYZus{}name} \PY{o}{=} \PY{l+m+mi}{10}
         \PY{n}{loser\PYZus{}player\PYZus{}id} \PY{o}{=} \PY{l+m+mi}{11}
         \PY{n}{loser\PYZus{}slug} \PY{o}{=} \PY{l+m+mi}{12}
         \PY{n}{winner\PYZus{}seed} \PY{o}{=} \PY{l+m+mi}{13}
         \PY{n}{loser\PYZus{}seed} \PY{o}{=} \PY{l+m+mi}{14}
         \PY{n}{match\PYZus{}score\PYZus{}tiebreaks} \PY{o}{=} \PY{l+m+mi}{15}
         \PY{n}{winner\PYZus{}sets\PYZus{}won} \PY{o}{=} \PY{l+m+mi}{16}
         \PY{n}{loser\PYZus{}sets\PYZus{}won} \PY{o}{=} \PY{l+m+mi}{17}
         \PY{n}{winner\PYZus{}games\PYZus{}won} \PY{o}{=} \PY{l+m+mi}{18}
         \PY{n}{loser\PYZus{}games\PYZus{}won} \PY{o}{=} \PY{l+m+mi}{19}
         \PY{n}{winner\PYZus{}tiebreaks\PYZus{}won} \PY{o}{=} \PY{l+m+mi}{20}
         \PY{n}{loser\PYZus{}tiebreaks\PYZus{}won} \PY{o}{=} \PY{l+m+mi}{21}
         \PY{n}{match\PYZus{}id} \PY{o}{=} \PY{l+m+mi}{22}
         \PY{n}{match\PYZus{}stats\PYZus{}url\PYZus{}suffix} \PY{o}{=} \PY{l+m+mi}{23}
         
         \PY{c+c1}{\PYZsh{}labels de matchstats}
         \PY{n}{ms\PYZus{}tourney\PYZus{}order} \PY{o}{=} \PY{l+m+mi}{0}
         \PY{n}{ms\PYZus{}match\PYZus{}id} \PY{o}{=} \PY{l+m+mi}{1}
         \PY{n}{ms\PYZus{}match\PYZus{}stats\PYZus{}url\PYZus{}suffix} \PY{o}{=} \PY{l+m+mi}{2}
         \PY{n}{ms\PYZus{}match\PYZus{}time} \PY{o}{=} \PY{l+m+mi}{3}
         \PY{n}{ms\PYZus{}match\PYZus{}duration} \PY{o}{=} \PY{l+m+mi}{4}
         \PY{n}{ms\PYZus{}winner\PYZus{}aces} \PY{o}{=} \PY{l+m+mi}{5}
         \PY{n}{ms\PYZus{}winner\PYZus{}double\PYZus{}faults} \PY{o}{=} \PY{l+m+mi}{6}
         \PY{n}{ms\PYZus{}winner\PYZus{}first\PYZus{}serves\PYZus{}in} \PY{o}{=} \PY{l+m+mi}{7}
         \PY{n}{ms\PYZus{}winner\PYZus{}first\PYZus{}serves\PYZus{}total} \PY{o}{=} \PY{l+m+mi}{8}
         \PY{n}{ms\PYZus{}winner\PYZus{}first\PYZus{}serve\PYZus{}points\PYZus{}won} \PY{o}{=} \PY{l+m+mi}{9}
         \PY{n}{ms\PYZus{}winner\PYZus{}first\PYZus{}serve\PYZus{}points\PYZus{}total} \PY{o}{=} \PY{l+m+mi}{10}
         \PY{n}{ms\PYZus{}winner\PYZus{}second\PYZus{}serve\PYZus{}points\PYZus{}won} \PY{o}{=} \PY{l+m+mi}{11}
         \PY{n}{ms\PYZus{}winner\PYZus{}second\PYZus{}serve\PYZus{}points\PYZus{}total} \PY{o}{=} \PY{l+m+mi}{12}
         \PY{n}{ms\PYZus{}winner\PYZus{}break\PYZus{}points\PYZus{}saved} \PY{o}{=} \PY{l+m+mi}{13}
         \PY{n}{ms\PYZus{}winner\PYZus{}break\PYZus{}points\PYZus{}serve\PYZus{}total} \PY{o}{=} \PY{l+m+mi}{14}
         \PY{n}{ms\PYZus{}winner\PYZus{}service\PYZus{}points\PYZus{}won} \PY{o}{=} \PY{l+m+mi}{15}
         \PY{n}{ms\PYZus{}winner\PYZus{}service\PYZus{}points\PYZus{}total} \PY{o}{=} \PY{l+m+mi}{16}
         \PY{n}{ms\PYZus{}winner\PYZus{}first\PYZus{}serve\PYZus{}return\PYZus{}won} \PY{o}{=} \PY{l+m+mi}{17}
         \PY{n}{ms\PYZus{}winner\PYZus{}first\PYZus{}serve\PYZus{}return\PYZus{}total} \PY{o}{=} \PY{l+m+mi}{18}
         \PY{n}{ms\PYZus{}winner\PYZus{}second\PYZus{}serve\PYZus{}return\PYZus{}won} \PY{o}{=} \PY{l+m+mi}{19}
         \PY{n}{ms\PYZus{}winner\PYZus{}second\PYZus{}serve\PYZus{}return\PYZus{}total} \PY{o}{=} \PY{l+m+mi}{20}
         \PY{n}{ms\PYZus{}winner\PYZus{}break\PYZus{}points\PYZus{}converted} \PY{o}{=} \PY{l+m+mi}{21}
         \PY{n}{ms\PYZus{}winner\PYZus{}break\PYZus{}points\PYZus{}return\PYZus{}total} \PY{o}{=} \PY{l+m+mi}{22}
         \PY{n}{ms\PYZus{}winner\PYZus{}service\PYZus{}games\PYZus{}played} \PY{o}{=} \PY{l+m+mi}{23}
         \PY{n}{ms\PYZus{}winner\PYZus{}return\PYZus{}games\PYZus{}played} \PY{o}{=} \PY{l+m+mi}{24}
         \PY{n}{ms\PYZus{}winner\PYZus{}return\PYZus{}points\PYZus{}won} \PY{o}{=} \PY{l+m+mi}{25}
         \PY{n}{ms\PYZus{}winner\PYZus{}return\PYZus{}points\PYZus{}total} \PY{o}{=} \PY{l+m+mi}{26}
         \PY{n}{ms\PYZus{}winner\PYZus{}total\PYZus{}points\PYZus{}won} \PY{o}{=} \PY{l+m+mi}{27}
         \PY{n}{ms\PYZus{}winner\PYZus{}total\PYZus{}points\PYZus{}total} \PY{o}{=} \PY{l+m+mi}{28}
         \PY{n}{ms\PYZus{}loser\PYZus{}aces} \PY{o}{=} \PY{l+m+mi}{29}
         \PY{n}{ms\PYZus{}loser\PYZus{}double\PYZus{}faults} \PY{o}{=} \PY{l+m+mi}{30}
         \PY{n}{ms\PYZus{}loser\PYZus{}first\PYZus{}serves\PYZus{}in} \PY{o}{=} \PY{l+m+mi}{31}
         \PY{n}{ms\PYZus{}loser\PYZus{}first\PYZus{}serves\PYZus{}total} \PY{o}{=} \PY{l+m+mi}{32}
         \PY{n}{ms\PYZus{}loser\PYZus{}first\PYZus{}serve\PYZus{}points\PYZus{}won} \PY{o}{=} \PY{l+m+mi}{33}
         \PY{n}{ms\PYZus{}loser\PYZus{}first\PYZus{}serve\PYZus{}points\PYZus{}total} \PY{o}{=} \PY{l+m+mi}{34}
         \PY{n}{ms\PYZus{}loser\PYZus{}second\PYZus{}serve\PYZus{}points\PYZus{}won} \PY{o}{=} \PY{l+m+mi}{35}
         \PY{n}{ms\PYZus{}loser\PYZus{}second\PYZus{}serve\PYZus{}points\PYZus{}total} \PY{o}{=} \PY{l+m+mi}{36}
         \PY{n}{ms\PYZus{}loser\PYZus{}break\PYZus{}points\PYZus{}saved} \PY{o}{=} \PY{l+m+mi}{37}
         \PY{n}{ms\PYZus{}loser\PYZus{}break\PYZus{}points\PYZus{}serve\PYZus{}total} \PY{o}{=} \PY{l+m+mi}{38}
         \PY{n}{ms\PYZus{}loser\PYZus{}service\PYZus{}points\PYZus{}won} \PY{o}{=} \PY{l+m+mi}{39}
         \PY{n}{ms\PYZus{}loser\PYZus{}service\PYZus{}points\PYZus{}total} \PY{o}{=} \PY{l+m+mi}{40}
         \PY{n}{ms\PYZus{}loser\PYZus{}first\PYZus{}serve\PYZus{}return\PYZus{}won} \PY{o}{=} \PY{l+m+mi}{41}
         \PY{n}{ms\PYZus{}loser\PYZus{}first\PYZus{}serve\PYZus{}return\PYZus{}total} \PY{o}{=} \PY{l+m+mi}{42}
         \PY{n}{ms\PYZus{}loser\PYZus{}second\PYZus{}serve\PYZus{}return\PYZus{}won} \PY{o}{=} \PY{l+m+mi}{43}
         \PY{n}{ms\PYZus{}loser\PYZus{}second\PYZus{}serve\PYZus{}return\PYZus{}total} \PY{o}{=} \PY{l+m+mi}{44}
         \PY{n}{ms\PYZus{}loser\PYZus{}break\PYZus{}points\PYZus{}converted} \PY{o}{=} \PY{l+m+mi}{45}
         \PY{n}{ms\PYZus{}loser\PYZus{}break\PYZus{}points\PYZus{}return\PYZus{}total} \PY{o}{=} \PY{l+m+mi}{46}
         \PY{n}{ms\PYZus{}loser\PYZus{}service\PYZus{}games\PYZus{}played} \PY{o}{=} \PY{l+m+mi}{47}
         \PY{n}{ms\PYZus{}loser\PYZus{}return\PYZus{}games\PYZus{}played} \PY{o}{=} \PY{l+m+mi}{48}
         \PY{n}{ms\PYZus{}loser\PYZus{}return\PYZus{}points\PYZus{}won} \PY{o}{=} \PY{l+m+mi}{49}
         \PY{n}{ms\PYZus{}loser\PYZus{}return\PYZus{}points\PYZus{}total} \PY{o}{=} \PY{l+m+mi}{50}
         \PY{n}{ms\PYZus{}loser\PYZus{}total\PYZus{}points\PYZus{}won} \PY{o}{=} \PY{l+m+mi}{51}
         \PY{n}{ms\PYZus{}loser\PYZus{}total\PYZus{}points\PYZus{}total} \PY{o}{=} \PY{l+m+mi}{52}
\end{Verbatim}


    \begin{Verbatim}[commandchars=\\\{\}]
{\color{incolor}In [{\color{incolor}108}]:} \PY{k+kn}{import} \PY{n+nn}{pandas} \PY{k}{as} \PY{n+nn}{pd}
          \PY{n}{pd\PYZus{}my\PYZus{}data\PYZus{}scores} \PY{o}{=} \PY{n}{pd}\PY{o}{.}\PY{n}{DataFrame}\PY{p}{(}\PY{n}{data}\PY{o}{=}\PY{n}{my\PYZus{}data\PYZus{}scores}\PY{p}{)}
          \PY{n}{pd\PYZus{}my\PYZus{}data\PYZus{}scores}\PY{o}{.}\PY{n}{columns}\PY{o}{=}\PY{p}{[}\PY{l+s+s1}{\PYZsq{}}\PY{l+s+s1}{tourney\PYZus{}year\PYZus{}id}\PY{l+s+s1}{\PYZsq{}}\PY{p}{,}\PY{l+s+s1}{\PYZsq{}}\PY{l+s+s1}{tourney\PYZus{}order}\PY{l+s+s1}{\PYZsq{}}\PY{p}{,}\PY{l+s+s1}{\PYZsq{}}\PY{l+s+s1}{tourney\PYZus{}slug}\PY{l+s+s1}{\PYZsq{}}\PY{p}{,}\PY{l+s+s1}{\PYZsq{}}\PY{l+s+s1}{tourney\PYZus{}url\PYZus{}suffix}\PY{l+s+s1}{\PYZsq{}}\PY{p}{,}\PY{l+s+s1}{\PYZsq{}}\PY{l+s+s1}{tourney\PYZus{}round\PYZus{}name}\PY{l+s+s1}{\PYZsq{}}\PY{p}{,}\PY{l+s+s1}{\PYZsq{}}\PY{l+s+s1}{round\PYZus{}order}\PY{l+s+s1}{\PYZsq{}}\PY{p}{,}\PY{l+s+s1}{\PYZsq{}}\PY{l+s+s1}{match\PYZus{}order}\PY{l+s+s1}{\PYZsq{}}\PY{p}{,}\PY{l+s+s1}{\PYZsq{}}\PY{l+s+s1}{winner\PYZus{}name}\PY{l+s+s1}{\PYZsq{}}\PY{p}{,}\PY{l+s+s1}{\PYZsq{}}\PY{l+s+s1}{winner\PYZus{}player\PYZus{}id}\PY{l+s+s1}{\PYZsq{}}\PY{p}{,}\PY{l+s+s1}{\PYZsq{}}\PY{l+s+s1}{winner\PYZus{}slug}\PY{l+s+s1}{\PYZsq{}}\PY{p}{,}\PY{l+s+s1}{\PYZsq{}}\PY{l+s+s1}{loser\PYZus{}name}\PY{l+s+s1}{\PYZsq{}}\PY{p}{,}\PY{l+s+s1}{\PYZsq{}}\PY{l+s+s1}{loser\PYZus{}player\PYZus{}id}\PY{l+s+s1}{\PYZsq{}}\PY{p}{,}\PY{l+s+s1}{\PYZsq{}}\PY{l+s+s1}{loser\PYZus{}slug}\PY{l+s+s1}{\PYZsq{}}\PY{p}{,}\PY{l+s+s1}{\PYZsq{}}\PY{l+s+s1}{winner\PYZus{}seed}\PY{l+s+s1}{\PYZsq{}}\PY{p}{,}\PY{l+s+s1}{\PYZsq{}}\PY{l+s+s1}{loser\PYZus{}seed}\PY{l+s+s1}{\PYZsq{}}\PY{p}{,}\PY{l+s+s1}{\PYZsq{}}\PY{l+s+s1}{match\PYZus{}score\PYZus{}tiebreaks}\PY{l+s+s1}{\PYZsq{}}\PY{p}{,}\PY{l+s+s1}{\PYZsq{}}\PY{l+s+s1}{winner\PYZus{}sets\PYZus{}won}\PY{l+s+s1}{\PYZsq{}}\PY{p}{,}\PY{l+s+s1}{\PYZsq{}}\PY{l+s+s1}{loser\PYZus{}sets\PYZus{}won}\PY{l+s+s1}{\PYZsq{}}\PY{p}{,}\PY{l+s+s1}{\PYZsq{}}\PY{l+s+s1}{winner\PYZus{}games\PYZus{}won}\PY{l+s+s1}{\PYZsq{}}\PY{p}{,}\PY{l+s+s1}{\PYZsq{}}\PY{l+s+s1}{loser\PYZus{}games\PYZus{}won}\PY{l+s+s1}{\PYZsq{}}\PY{p}{,}\PY{l+s+s1}{\PYZsq{}}\PY{l+s+s1}{winner\PYZus{}tiebreaks\PYZus{}won}\PY{l+s+s1}{\PYZsq{}}\PY{p}{,}\PY{l+s+s1}{\PYZsq{}}\PY{l+s+s1}{loser\PYZus{}tiebreaks\PYZus{}won}\PY{l+s+s1}{\PYZsq{}}\PY{p}{,}\PY{l+s+s1}{\PYZsq{}}\PY{l+s+s1}{match\PYZus{}id}\PY{l+s+s1}{\PYZsq{}}\PY{p}{,}\PY{l+s+s1}{\PYZsq{}}\PY{l+s+s1}{match\PYZus{}stats\PYZus{}url\PYZus{}suffix}\PY{l+s+s1}{\PYZsq{}}\PY{p}{]}
\end{Verbatim}


    Debajo se muestra un ejemplo del dataset scores obtenido:

    \begin{Verbatim}[commandchars=\\\{\}]
{\color{incolor}In [{\color{incolor}110}]:} \PY{n}{pd\PYZus{}my\PYZus{}data\PYZus{}scores}\PY{o}{.}\PY{n}{loc}\PY{p}{[}\PY{p}{[}\PY{l+m+mi}{0}\PY{p}{,}\PY{l+m+mi}{1}\PY{p}{,}\PY{l+m+mi}{2}\PY{p}{,}\PY{l+m+mi}{3}\PY{p}{,}\PY{l+m+mi}{4}\PY{p}{,}\PY{l+m+mi}{5}\PY{p}{]}\PY{p}{]}
\end{Verbatim}


\begin{Verbatim}[commandchars=\\\{\}]
{\color{outcolor}Out[{\color{outcolor}110}]:}   tourney\_year\_id  tourney\_order tourney\_slug  \textbackslash{}
          0    b'1991-7308'              1  b'adelaide'   
          1    b'1991-7308'              1  b'adelaide'   
          2    b'1991-7308'              1  b'adelaide'   
          3    b'1991-7308'              1  b'adelaide'   
          4    b'1991-7308'              1  b'adelaide'   
          5    b'1991-7308'              1  b'adelaide'   
          
                                           tourney\_url\_suffix tourney\_round\_name  \textbackslash{}
          0  b'/en/scores/archive/adelaide/7308/1991/results'          b'Finals'   
          1  b'/en/scores/archive/adelaide/7308/1991/results'     b'Semi-Finals'   
          2  b'/en/scores/archive/adelaide/7308/1991/results'     b'Semi-Finals'   
          3  b'/en/scores/archive/adelaide/7308/1991/results'  b'Quarter-Finals'   
          4  b'/en/scores/archive/adelaide/7308/1991/results'  b'Quarter-Finals'   
          5  b'/en/scores/archive/adelaide/7308/1991/results'  b'Quarter-Finals'   
          
             round\_order  match\_order       winner\_name winner\_player\_id  \textbackslash{}
          0            1            1  b'Nicklas Kulti'          b'k181'   
          1            2            1  b'Michael Stich'          b's351'   
          2            2            2  b'Nicklas Kulti'          b'k181'   
          3            3            1    b'Jim Courier'          b'c243'   
          4            3            2  b'Michael Stich'          b's351'   
          5            3            3  b'Nicklas Kulti'          b'k181'   
          
                  winner\_slug                    {\ldots}                     loser\_seed  \textbackslash{}
          0  b'nicklas-kulti'                    {\ldots}                           b'6'   
          1  b'michael-stich'                    {\ldots}                           b'2'   
          2  b'nicklas-kulti'                    {\ldots}                            b''   
          3    b'jim-courier'                    {\ldots}                           b'Q'   
          4  b'michael-stich'                    {\ldots}                            b''   
          5  b'nicklas-kulti'                    {\ldots}                            b''   
          
            match\_score\_tiebreaks winner\_sets\_won loser\_sets\_won winner\_games\_won  \textbackslash{}
          0           b'63 16 62'               2              1               13   
          1           b'64 76(6)'               2              0               13   
          2              b'75 64'               2              0               13   
          3           b'76(3) 63'               2              0               13   
          4           b'36 63 63'               2              1               15   
          5              b'63 60'               2              0               12   
          
            loser\_games\_won  winner\_tiebreaks\_won  loser\_tiebreaks\_won  \textbackslash{}
          0              11                     0                    0   
          1              10                     1                    0   
          2               9                     0                    0   
          3               9                     1                    0   
          4              12                     0                    0   
          5               3                     0                    0   
          
                           match\_id                     match\_stats\_url\_suffix  
          0  b'1991-7308-k181-s351'  b'/en/scores/1991/7308/MS001/match-stats'  
          1  b'1991-7308-s351-c243'  b'/en/scores/1991/7308/MS003/match-stats'  
          2  b'1991-7308-k181-l206'  b'/en/scores/1991/7308/MS002/match-stats'  
          3  b'1991-7308-c243-s367'  b'/en/scores/1991/7308/MS007/match-stats'  
          4  b'1991-7308-s351-a031'  b'/en/scores/1991/7308/MS006/match-stats'  
          5  b'1991-7308-k181-s424'  b'/en/scores/1991/7308/MS005/match-stats'  
          
          [6 rows x 24 columns]
\end{Verbatim}
            
    Los datos luego serán seleccionados partido a partido, a partir del
dataset obtenido. Se realiza la siguiente división:

\begin{longtable}[]{@{}ll@{}}
\toprule
Categoría & Detalle\tabularnewline
\midrule
\endhead
ID1 & ID Jugador 1\tabularnewline
ID2 & ID Jugador 2\tabularnewline
RANK & Ranking ATP\tabularnewline
POINTS & Puntos ATP\tabularnewline
FS & Porcentaje de primeros servicios\tabularnewline
W1SP & Porcentaje de primeros servicios ganados\tabularnewline
W2SP & Porcentaje de segundos servicios ganados\tabularnewline
WSP & Porcentaje de servicios ganados globales\tabularnewline
WRP & Porcentaje de devolución ganada\tabularnewline
TPW & Porcentaje de puntos totales ganados\tabularnewline
TMW & Porcentaje de todos los partidos ganados\tabularnewline
ACES & Promedio de aces por game\tabularnewline
DF & Promedio de doble faltas por game\tabularnewline
UE & Promedio de errores no forzados por game\tabularnewline
WIS & Promedio de winners por game\tabularnewline
BP & Porcentaje de break points ganados\tabularnewline
NA & Porcentaje de puntos en la red ganados\tabularnewline
A1S & Promedio de velocidad primer servicio\tabularnewline
A2S & Promedio de velocidad de segundo servicio\tabularnewline
FATIGUE & Fatiga por partidos jugados en 3 dias
anteriores\tabularnewline
RETIRED & Indica si es el primer partido luego de un
retiro\tabularnewline
DIRECT & Ventaja H2H entre los jugadores\tabularnewline
\bottomrule
\end{longtable}

Cada partido tiene estas características que suelen indicar las
diferencias entre ambos jugadores. Por lo que por ejemplo, para ATPRank,
si RANK1 es el ranking del jugador 1 y RANK2 es el ranking del jugador
2, RANK = RANK1 - RANK2. Para el entrenamiento, cada partido tendrá como
salida un vector que inidque que jugador gana o que jugador pierde. Es
decir, 1 si el jugador ganó o 0 si el jugador perdió.

Finalmente, el dataset será dividido en tres partes: - Set de
entrenamiento (2008 - 2014) - Set de validación (2015 - 2016) - Set de
prueba (2017-2018)

    \subsubsection{Soluciones Propuestas}\label{soluciones-propuestas}

Se propone evaluar si utilizar una red multicapa en el entorno
tensorflow para la implementación del proyecto (el cual se entrena con
todo el dataset) o una red neuronal recurrente la cual posee una memoria
'selectiva' de partidos anteriores. Estas soluciones están sujetas a
cambiar de acuerdo a las nuevas técnicas a aprender en la materia.

\subsubsection{Medición de la Calidad de la Solución
Obtenida}\label{mediciuxf3n-de-la-calidad-de-la-soluciuxf3n-obtenida}

La solución obtenida se limitará a predecir el ganador de cada partido.
Esto tiene como consecuencia determinar el resultado de cada uno de los
torneos a nivel ATP de la temporada, como así también el resultado del
ranking al finalizar la temporada. Se contrastaran con los resultados
reales, pretendiéndose como mínimo una efectividad superior al 50\%.

\subsubsection{Proyectos Similares Implementados en
Internet}\label{proyectos-similares-implementados-en-internet}

El proyecto de referencia será la tesis presente en el siguiente link:
https://www.doc.ic.ac.uk/teaching/distinguished-projects/2015/m.sipko.pdf
el cuál plantea la predicción de partidos de tenis con diversos métodos
incluyendo Machine Learning y mostrando sus resultados. Otro proyecto
más breve se encuentra en el siguiente link:
http://deepakn94.github.io/assets/papers/6.867.pdf. La técnica para
solucionar el problema también puede referirse a predictores de otros
deportes.


    % Add a bibliography block to the postdoc
    
    
    
    \end{document}
